\section{Next steps / Outlook}

This chapter presents the next steps of our project and explores possibilities 
for enhancement with additional time and resources. 
We will outline specific options and solutions to optimize and advance project outcomes. 
Through this exploration, we aim to maximize the full potential of our project 
for maximum impact and success.

\subsection{Expanding from One Locker Prototype to Locker Systems}

Currently, our system features a single locker prototype designed to demonstrate the functionality and implementation of our technology. This prototype serves as a proof of concept for showcasing our capabilities.

Moving forward, our objective is to scale our system by implementing a comprehensive locker system with customizable configurations to meet diverse needs. For example:

\begin{itemize}
    \item \textbf{Thinner Lockers for Laptops}: We plan to introduce lockers with wider compartments suitable for securely storing laptops and larger electronic devices.
    \item \textbf{Smaller Lockers for Phones and Small Items}: Additionally, we aim to offer compact lockers designed specifically for storing mobile phones, wallets, keys, and other smaller items.
    \item \textbf{Variable Compartment Sizes}: Our locker system will feature adjustable compartment sizes to accommodate various items, providing flexibility for different storage requirements.
\end{itemize}

This transition from a single prototype to a versatile locker system infrastructure represents a significant advancement in our capabilities. It enables us to cater to a wide range of applications, including secure storage solutions tailored to specific customer needs and use cases.


\subsection{Real-time Charging Level Monitoring}
The current approach involves utilizing an Arduino Current Power Sensor to measure the
current amperage flowing into the device, 
enabling determination of the charging status and current amperage. 
This method provides basic charging information but lacks insight into the battery's state of charge.

To enhance this system, our objective is to implement a more sophisticated feature that enables 
real-time monitoring of the device's charging percentage. This enhancement aims to facilitate smart and 
sustainable charging practices to mitigate premature battery degradation.


\subsection{Implementation of QR Codes for Streamlined Inventory Management}
Currently, the inventory management system relies on manual entry through an administrative interface,
 where item parameters must be manually inputted to add items to the system. 
 Searching for specific items within the database requires querying by ID, name, or other attributes.

Our proposed approach involves transitioning to a QR code-based system, where each item is assigned
 a unique QR code identifier. 
 This implementation aims to simplify inventory management by enabling rapid item identification 
 through QR code scanning, streamlining the process for administrators to locate and manage items
 within the system.

\subsection{Scheduled Maintenance Protocol}
As our system continues to grow and evolve, particularly with the increasing complexity of our inventory management,
we recognize the critical importance of ensuring reliability and performance. 
To achieve this, we are dedicated to establishing a structured and scheduled maintenance protocol. 
This protocol will enable us to proactively manage maintenance activities, optimize asset performance,
and minimize unplanned downtime, specifically targeting our inventory items. By implementing this approach,
we aim to enhance overall system reliability, support scalability,
and ensure the continuous functionality and efficiency of our inventory management processes as we expand.


\subsection{Automated Reservation System}


