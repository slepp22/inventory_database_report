\section{Methodology}
{\tiny Written by: Felix}\\

\begin{enumerate}
    \item which are the steps that are taken 
    in which order to reach the result
    \item which steps are taken
    \item which tools were used (keep this short)
    \item always give reasons for WHY you chose 
    certain steps
    \item always give reasons for WHY you 
    chose certain tools
\end{enumerate}

In this section, we outline the methodology used for the project, covering planning, design, development, integration, and deployment phases.

\subsection{Planning Phase}

Effective planning is crucial for the success of our project, especially considering our unique work arrangement. With only one week of in-person collaboration and the majority of our work conducted online, meticulous planning becomes essential to ensure smooth coordination, clear communication, and efficient task management. This phase lays the groundwork for our project's success by defining roles, establishing communication channels, and outlining our approach to project execution. By emphasizing thorough planning, we aim to maximize productivity, maintain team cohesion, and overcome the challenges posed by our distributed work environment.

\begin{enumerate}
    \item \textbf{Team Building}:
    \begin{itemize}
        \item Formed a collaborative and diverse team through spontaneous selection based on available participants. The team consisted of individuals from various international backgrounds, including Finnish \& German students and other participants, fostering a multicultural and multidisciplinary environment.
        \item Leveraged the diverse composition of the team to promote cross-cultural perspectives and interdisciplinary collaboration, enhancing creativity and problem-solving capabilities during project execution.
    \end{itemize}
    
    \item \textbf{Project Selection}:
    \begin{itemize}
        \item Engaged in a facilitated process where a professor from Centria University presented a range of project topics to the team. Each topic was evaluated based on criteria such as feasibility, innovation potential, and alignment with organizational goals.
        \item Collaboratively deliberated and selected the most promising project topic from the options provided, considering the team's capabilities and interests.
    \end{itemize}
    \textbf{Why we decided for project}
    \begin{enumerate}
        \item The team expressed a strong interest and enthusiasm for topics related to databases, frontend development, and locker systems, reflecting our collective passion and desire to deepen our expertise in these areas.
        \item Recognizing the importance of specialization and professional growth, we chose these topics to leverage our existing skills and gain hands-on experience in critical aspects of modern software development.
        \item By focusing on database management, frontend design, and locker system implementation, we aimed to enhance our proficiency and readiness to tackle real-world challenges in these specialized domains.
        \item Our commitment to professionalism motivated us to select topics that align with our career aspirations and provide valuable learning opportunities to develop industry-relevant skills.
        \item Additionally, we identified strong future perspectives and potential career opportunities associated with database management, frontend development, and smart locker systems, making these topics strategic choices for our educational and professional growth.
    \end{enumerate}
    

    \item \textbf{Project Brainstorm}:
    \begin{itemize}
        \item Conducted facilitated brainstorming sessions to generate creative ideas and identify key project features and requirements based on stakeholder inputs.
        \item \textbf{Why we conducted brainstorming sessions}:
            \begin{enumerate}
                \item Emphasized inclusivity and openness to ensure all team members could freely contribute their ideas without constraints.
                \item Focused on capturing a wide range of ideas and perspectives to explore innovative solutions and address diverse project requirements.
                \item Encouraged collaborative idea generation to foster team cohesion and collective ownership of project concepts.
                \item Recognized the importance of comprehensive planning and idea exploration to minimize the risk of overlooking critical aspects of the project.
            \end{enumerate}
    \end{itemize}



    \item \textbf{Organizational Structure}:
    \begin{itemize}
        \item Defined clear roles and responsibilities within the team and established communication channels to ensure effective collaboration and coordination throughout the project lifecycle.
        \item \textbf{Why we defined the organizational structure}:
            \begin{enumerate}
                \item Facilitated clear understanding of individual roles and responsibilities, promoting accountability and ownership within the team.
                \item Established efficient communication channels and workflows to facilitate seamless collaboration and coordination among team members.
                \item Enhanced project management and coordination by assigning specific tasks and defining decision-making processes.
                \item Ensured smooth project execution by clarifying who organizes meetings, manages tasks, and facilitates team interactions.
            \end{enumerate}
            \item \textbf{Communication Tools Used}:
            \begin{enumerate}
                \item \textbf{ClickUp (Kanban Board)}: Implemented ClickUp as a Kanban board tool to manage tasks, track progress, and visualize workflow stages, enabling efficient project planning and task management.
                \item \textbf{GitHub}: Utilized GitHub for version control and collaboration on project repositories, allowing team members to share and synchronize code, track changes, and manage project documentation.
                \item \textbf{Discord}: Utilized Discord as a primary platform for organizing project discussions, sharing updates, and facilitating real-time communication among team members.
                \item \textbf{WhatsApp}: Leveraged WhatsApp for conducting polls, scheduling meetings, and exchanging short messages for quick updates and coordination.
                \item \textbf{Google Calendar}: Used Google Calendar as a centralized platform for scheduling meetings and appointments, ensuring consistency across different time zones and facilitating team coordination.
                \item \textbf{Google Drive}: Utilized Google Drive for organizing presentations, documents, and images, providing a collaborative workspace for sharing and accessing project-related resources.
            \end{enumerate}        
    \end{itemize}
\end{enumerate}


\subsection{Design Phase}

\begin{enumerate}
    \item \textbf{Personas}:
    \begin{itemize}
        \item Developed user personas based on target audience demographics, behaviors, and needs to guide design decisions and prioritize features.
        \item \textbf{Why we developed user personas}:
            \begin{enumerate}
                \item Gained insights into the preferences, behaviors, and needs of our target audience to inform design decisions and feature prioritization.
                \item Enabled a user-centered design approach by creating fictional representations of potential users, ensuring that our solutions address real-world user requirements.
                \item Facilitated empathy and understanding among team members by visualizing and empathizing with user personas, enhancing our ability to design intuitive and user-friendly interfaces.
                \item Supported stakeholder engagement and decision-making processes by aligning design choices with identified user preferences and pain points.
            \end{enumerate}
    \end{itemize}

    
    \item \textbf{Use Cases}:
    \begin{itemize}
        \item Defined use cases to capture interactions between users and the system, facilitating a clear understanding of functional requirements and user workflows.
        \item \textbf{Why we defined use cases}:
            \begin{enumerate}
                \item Enhanced clarity and specificity in defining system functionalities and user interactions, ensuring alignment with project objectives.
                \item Enabled systematic validation of system behavior against user requirements, identifying potential mistakes and gaps in functionality.
                \item Provided a clear and structured representation of user workflows, guiding the design and development process to meet user expectations.
                \item Supported effective communication among team members and stakeholders by visualizing user-system interactions and functional requirements.
            \end{enumerate}
    \end{itemize}

    \item \textbf{Wireframes}:
    \begin{itemize}
        \item Created wireframes using \textbf{Miro} to visualize the user interface layout and navigation structure, allowing for early feedback and iteration on design concepts.
        \item \textbf{Why we created wireframes}:
            \begin{enumerate}
                \item Facilitated the rapid prototyping of frontend design concepts, enabling quick visualization and validation of UI layout and navigation.
                \item Provided a tangible representation of the user interface, allowing for early feedback and iteration to refine design concepts and improve usability.
                \item Supported user testing activities by presenting a preliminary version of the UI, enabling stakeholders and end-users to provide valuable feedback.
                \item Enhanced collaboration among team members and stakeholders by aligning design expectations and ensuring a shared vision of the final product.
            \end{enumerate}
    \end{itemize}
    
    \item \textbf{Wireframe Tool Used}:
        \begin{itemize}
            \item \textbf{Miro (Wireframing)}: Employed Miro for wireframing due to its ease of use and ability to deliver fast results. Miro's online platform facilitated seamless sharing and collaboration among team members, enhancing creativity and enabling remote collaboration on design concepts and workflows.
        \end{itemize}


    \item \textbf{Software Architecture}:
    \begin{itemize}
            \item Designed a scalable and modular software architecture to support future expansion and maintenance of the system.       
    \item \textbf{Architecture Tools Used}:
        \begin{enumerate}
            \item \textbf{Frontend: Tailwind CSS and Vue.js}: Employed Tailwind CSS and Vue.js for frontend development. Tailwind CSS offers utility-first styling, facilitating rapid UI development, while Vue.js provides a reactive framework for building interactive user interfaces.
            \item \textbf{Backend: FASTAPI, SQLAlchemy, Docker}: Implemented the backend using FASTAPI for creating RESTful APIs, SQLAlchemy for database interaction, and Docker for containerization. FASTAPI's asynchronous capabilities and SQLAlchemy's ORM simplifies backend development, while Docker ensures easy deployment and scalability.
            \item \textbf{Hardware Sensors (Refer to Section \ref{sec:ArduinoSensors})}: Utilized various sensors as detailed in Section 7.5.4 for hardware integration within the system, enabling data collection and interaction with physical components.
        \end{enumerate}
    \end{itemize}
    
    
    \item \textbf{3D Modeling}:
    \begin{itemize}
            \item Created 3D models using \textbf{Tinkercad} to visualize physical components and interactions within the system, aiding in the development of prototypes and simulations.
        \item \textbf{3D Modeling Tools Used}:
        \begin{enumerate}
            \item \textbf{Tinkercad \cite{tinkercad}}: Selected Tinkercad as our 3D modeling tool for its intuitive interface and online accessibility, making it ideal for team members new to 3D modeling. The tool's manageable learning curve allowed quick understanding of 3D modeling concepts, facilitating efficient creation of prototypes and simulations. Additionally, Tinkercad's online platform promoted seamless collaboration and sharing of 3D models among team members, enhancing teamwork and productivity.
        \end{enumerate}
    \end{itemize}
\end{enumerate}
   

\subsection{Development Phase}

\begin{enumerate}

    \item \textbf{Implementation}:
    \begin{itemize}
        \item Implemented core features and functionalities based on design specifications, utilizing appropriate programming languages and frameworks.
        \item \textbf{Technologies Used}:
            \begin{enumerate}
                \item \textbf{Frontend (Vue.js)}: Started frontend implementation using Vue.js, a progressive JavaScript framework for building user interfaces. Vue.js was chosen for its simplicity, reactivity, and component-based architecture, allowing rapid development of interactive frontend components.
                \item \textbf{Backend (Python with FASTAPI)}: Developed the backend using Python with FASTAPI, a modern web framework for building APIs with asynchronous support. FASTAPI's performance and ease of use were instrumental in creating scalable backend services.
                \item \textbf{Database (PostgreSQL with SQLAlchemy)}: Integrated PostgreSQL as the database management system, leveraging SQLAlchemy as the ORM (Object-Relational Mapping) tool. PostgreSQL was chosen for its reliability, SQL compliance, and seamless integration with SQLAlchemy, simplifying data modeling and interaction within the application.
                \item \textbf{Hardware (Arduino)}: Initiated hardware development using Arduino, an open-source electronics platform, for integrating sensors and controlling physical components. Arduino was selected based on team experience and the availability of hardware, facilitating familiarity and enabling efficient prototyping and sensor integration.
            \end{enumerate}
    \end{itemize}


    \item \textbf{Bi-weekly Meetings}:
    \begin{itemize}
        \item Conducted regular bi-weekly meetings, modeled after stand-up meetings, to review project progress, address challenges, and align project goals with stakeholders.
        \item \textbf{Meeting Format}:
            \begin{enumerate}
                \item Team members provided updates based on Kanban board tasks:
                    \begin{itemize}
                        \item What they accomplished since the last meeting.
                        \item Current tasks they are working on.
                        \item Planned tasks for the upcoming period.
                        \item Challenges or blockers they are facing.
                    \end{itemize}
                \item Encouraged problem-solving discussions to address challenges collaboratively within the team.
            \end{enumerate}
        \item \textbf{Communication Tools Used}:
            \begin{enumerate}
                \item \textbf{Google Meet or Discord}: Leveraged Google Meet or Discord as the primary platforms for conducting bi-weekly meetings. These tools facilitated real-time communication, screen sharing, and video conferencing, enabling effective discussions and updates among team members located remotely.
            \end{enumerate}
    \end{itemize}


 \item \textbf{Research and Development (R\&D)}:
    \begin{itemize}
        \item Invested time in R\&D to explore innovative solutions and technologies that could enhance project outcomes and performance.
        \item \textbf{R\&D Initiatives}:
            \begin{enumerate}
                \item Explored methods for displaying device battery status, ultimately opting for an intermediate solution that indicates power flow through the cable. This approach addressed immediate needs while paving the way for more sophisticated battery monitoring solutions in the future.
                \item Conducted research on deploying frontend and backend components, leveraging Google as a key tool for accessing documentation, tutorials, and best practices. Google facilitated the acquisition of valuable insights and guidance, aiding in the successful deployment of project components.
            \end{enumerate}
    \end{itemize}

    
\item \textbf{Building Prototype}:
    \begin{itemize}
        \item Developed functional prototypes to validate design assumptions that we can start installing the sensors with the Deployment Phase.
    
    \item \textbf{3D Printing Tool}:
    \begin{enumerate}
        \item Used the Anycubic Kobra 3D \citet{anycubic-kobra} printer to manufacture prototypes, enabling the installation of various sensors and components to support design validation and iterative development.
    \end{enumerate}
    \end{itemize}
\end{enumerate}



\subsection{Integration Phase}

\begin{enumerate}
    \item \textbf{Component Integration}:
    \begin{itemize}
        \item Integrated individual components and subsystems to ensure seamless interoperability and overall system functionality. Testing included using Postman for FastAPI, manual testing for Arduino components, and SonarLint for frontend code analysis.
    \end{itemize}

\item \textbf{Usability Test}:
    \begin{itemize}
        \item Conducted usability testing sessions on the frontend to evaluate design and usability aspects. Results from these tests informed UX improvements for enhanced user experience.
    \end{itemize}

\item \textbf{Integration Test}:
    \begin{itemize}
        \item Performed integration testing to validate end-to-end system functionality and identify compatibility issues. Leveraged automated tests with Postman for API interactions to ensure components work together seamlessly.
    \end{itemize}

\end{enumerate}



\subsection{Deployment Phase}

\begin{enumerate}
    \item \textbf{Frontend Deployment}:
        \begin{itemize}
            \item Deployed frontend components of the system to web servers or cloud platforms to make the user interface accessible to end-users. We chose Netlify for frontend deployment due to its simplicity and seamless integration with Git, enabling continuous deployment and easy updates of the user interface.
        \end{itemize}
        Detailed information on the frontend deployment with Netlify can be found in Section \ref{sec:Frontend_Implementation} 

    \item \textbf{Backend Deployment}:
        \begin{itemize}
            \item Deployed backend services and APIs to support the application logic and data processing tasks. The backend, including the database system, was deployed using Google Cloud Run solutions. Google Cloud Run's serverless architecture allowed for scalable and efficient deployment of backend components without managing infrastructure overhead.
        \end{itemize}
        Detailed information on the backend deployment with Google Cloud Run can be found in Section \ref{sec:Implementation_Backend}
\end{enumerate}

This methodology provided a structured approach to project planning, design, development, integration, and deployment, ensuring efficient execution and successful delivery of the project objectives.
