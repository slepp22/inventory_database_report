\subsubsection{Areas for Improvement}

While the backend implementation of the locker system prototype adheres to common best practices, several areas require further attention and refinement before the system can be considered production-ready:

\begin{itemize}
    \item \textbf{Error Handling}: The current implementation lacks comprehensive error handling mechanisms, which are essential for gracefully managing unexpected scenarios and providing informative feedback to users. For example, enhancing error responses with appropriate status codes and error messages would improve the system's robustness.

    \item \textbf{Infrastructure}: The prototype operates in a development environment and lacks the infrastructure necessary for deployment in a production setting. Implementing scalable infrastructure components, such as load balancers, auto-scaling groups, and fault-tolerant databases, is crucial for ensuring system reliability and performance under varying workloads.

    \item \textbf{Validation}: While basic input validation is incorporated into the API endpoints, there is room for strengthening validation logic to enforce data integrity and prevent malicious inputs. Implementing robust validation mechanisms, such as input sanitization and parameter validation, would enhance the security and reliability of the system.

    \item \textbf{Security}: Security measures, including authentication, authorization, and data encryption, are fundamental requirements for safeguarding sensitive information and protecting the system against security threats. Implementing authentication mechanisms, role-based access control (RBAC), and encryption protocols would mitigate security risks and enhance the system's trustworthiness.
\end{itemize}

Addressing these areas of improvement would contribute to the development of a robust, secure, and scalable backend system for the locker application.
