\section{Reflection and Lessons Learned}

\subsection{Successes and Effective Practices}

Reflecting on the international project to develop a locker system for universities, several valuable lessons were learned throughout the course of the project. These reflections include both the successes achieved and areas for improvement, providing insights that would guide future projects.

\subsubsection{Successes and Effective Practices}

Overall, the project concluded with a very satisfactory outcome for us, despite the challenges posed by the tight deadline. Key factors contributing to the success of the project include:

\begin{itemize}
    \item \textbf{Effective Communication}: Maintaining open and regular communication channels proved to be essential. Through platforms such as WhatsApp and Discord, along with bi-weekly meetings, team members remained well-informed about project progress and tasks.

    \item \textbf{Positive Team Dynamics}: The cohesive spirit within the team fostered a supportive environment where members easily assisted each other. This collaborative ethos not only enhanced productivity but also facilitated mutual learning and skill development.

    \item \textbf{Utilization of Kanban Board}: Adopting a Kanban board for task organization proved to be important in visualizing workflow and tracking individual and collective progress. The transparency afforded by this tool promoted accountability and facilitated efficient task allocation.

    \item \textbf{Adaptability to Remote Work}: Despite a preference for face-to-face collaboration, the team demonstrated adaptability in transitioning to online work modes when necessary. Effective utilization of online collaboration tools enabled seamless remote teamwork.

    \item \textbf{Skill Integration through Pair Programming}: Pair programming is a collaborative development technique where two programmers work together at one workstation. One programmer, known as the "driver," writes the code, while the other, the "navigator," reviews each line as it's typed. This approach encourages constant communication, instant feedback, and shared problem-solving.

    Pair programming was the right tool for our team for several reasons. Firstly, it facilitated knowledge sharing among team members with different expertise levels. Pairing individuals allowed for the transfer of skills and knowledge in real-time, fostering a supportive environment where team members could learn from each other's experiences.

    Moreover, pair programming promoted a deeper understanding of the codebase and project requirements. The constant dialogue between the driver and navigator ensured that code was thoroughly reviewed as it was written, reducing the likelihood of errors and enhancing overall code quality. Additionally, pairing individuals with varying skill levels encouraged less experienced team members to contribute actively while receiving guidance and mentorship from more experienced team members.
\end{itemize}

\subsection{Areas for Improvement and Lessons Learned}

While the project yielded very good results, several areas were identified for improvement, along with valuable lessons learned for future projects:

\begin{itemize}
    \item \textbf{Investment in Quality Components}: An incident involving the malfunction of inexpensive Arduino components underscored the importance of investing in quality hardware. In future projects, buying higher-quality components would mitigate risks of hardware failures and enhance system reliability.

    \item \textbf{Consideration of Hosting Platforms}: Initially opting for Amazon Web Services (AWS) for hosting, the project encountered unexpected costs associated with certain AWS components. Subsequently transitioning to Google Cloud required additional effort but provided valuable exposure to alternative cloud platforms.

    \item \textbf{Consolidation of API Documentation}: We generated API documentation in two separate documents, this led to discrepancies and minor bugs. Consolidating all API documentation into a single source of truth would streamline development processes and ensure consistency across the project.
\end{itemize}

In conclusion, the international project to develop a locker system for universities provided invaluable learning experiences and insights. While celebrating the successes achieved, it is important to acknowledge areas for improvement and incorporate lessons learned into future projects.